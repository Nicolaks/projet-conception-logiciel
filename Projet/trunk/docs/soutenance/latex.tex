  \documentclass[a4paper, 11pt]{article}
\usepackage[utf8]{inputenc} 
\usepackage[T1]{fontenc}
\usepackage{graphicx}
\usepackage[french]{babel} 
\usepackage{multirow,multicol}
\usepackage{amsmath, amssymb, latexsym}
\usepackage{pstricks,pst-node,pst-coil,pst-grad,pst-plot}
\usepackage{epsfig,subfigure}
\usepackage[lined,boxed]{algorithm}
\usepackage{algorithmic}
\usepackage{rotate}
\usepackage{url}
\usepackage{hyperref}
\usepackage{setspace}

\newtheorem{definition}{Definition}
\newtheorem{example}{Example}
\newtheorem{proposition}{Proposition}
\newtheorem{proof}{Proof}

\title{Présentation d'un Manic'shooter}
\author{Nicolas.A - David.R - Théo.B} 


\begin{document}

\maketitle
\tableofcontents
\section{Manic Shooter}
\subsection{Présentation-Objectif}

Présentation d'un Manic Shooter:
Manic shooter ou Shoot them up ou Shmup qui signifie littéralement "Descendez-les tous".
Un Manic shooter est un jeu ou le joueur doit diriger un personnage ou véhicule devant tuer un grand nombres d'ennemis à l'aide d'armes de plus en plus puissantes au fur et à mesure des niveaux, le personnages doit esquiver les tirs ou projectile ennemis.
Ce système de jeu est sorti en 1978 avec 
\href{http://dictionnaire.sensagent.leparisien.fr/Space%20Invaders/fr-fr/}{Space inverders}
 present dans les salles d'arcades de base c'est un jeu 2D il trouve son succès enfin des années 80 début 90 dès que le graphisme tri-dimensionnelle apparut, son succès disparut.

 \begin{figure*}[ht!]
 \centering
 \includegraphics[width=0.7\linewidth]{space.jpg}
 \caption{Space inverders}
 \label{fig::example::one}
\end{figure*}

\section{Architecture}


 \begin{figure*}[ht!]
 \centering
 \includegraphics[width=1.1\linewidth]{diagfinal.png}
 \caption{diagramme cas d'utilisation}
 \label{fig::example::one}
\end{figure*}

Ici nous avons a faire au premier menu qui va proposer de jouer , Quitter et de choisir la résolution de l'écran souhaitée.
Jouer lance pygame et donne accès au menu final du jeu ou lancer le jeu est possible ou de changer des paramètres comme les préférences de jeu du joueur.


\section{Fonctionnalité du jeu}


 	\subsection{Une carte}
 	
 	\begin{figure*}[ht!]
\centering
\includegraphics[width=0.8\linewidth]{Background.jpg}
\caption{Notre carte}
\end{figure*}
 	dans un premier temps il nous fallait une carte, c'est à dire un espace de jeu permettant a un héro d'être livrer à une épopée contre des ennemies,
 	notre premier réflexe à était de créer une grille et de créer un héro en l'initialisant en 1 et en initialisant des ennemies à 0 le reste de la grille étant que des "-", menant notre objectif jusqu'au bout nous nous sommes rendu compte qu'en associant l'interface graphique avec notre modèle de jeu posait problème et surtout qu'il existait une manière "plus simple" de faire la carte et surtout plus efficace, car en définissant la fenêtre avec Pygame cela aller être notre grille c'est à dire que la grille est la taille de fenêtre fois le nombre de pixels ce qui nous à simplifié la vie pour la suite.

 	
	\subsection{Un personnage}
un manic shooter à pour habitude d'avoir un héro qui ce déplace dans la carte, il nous fallait donc un héro ou personnage qui soit celui que le joueur va contrôler notre personnage :
\begin{figure*}[ht!]
\centering
\includegraphics[width=0.1\linewidth]{spaceCraft1.png}
\caption{Notre personnage trouver sur internet "open source"}
\end{figure*}
ce personnage est équiper d'un réacteur qui le suit il à fallut intégrer cela au déplacement du héro et de faire en sorte que le réacteur soit collé au héro donc que la position soit identique .... suite .....

	\subsection{Des ennemis}
Le héro n'est pas le seul personnage à géré il nous faut également gérer les ennemies
	\subsection{Déplacement}
Une fois le contexte poser pour les personnages / la carte, c'est maintenant qu'intervient la notion de déplacement évidement le héro doit se déplacer pour éviter les ennemies et avant dans sont épopée, mais les ennemies doivent ce déplacer également, il ce déplace aléatoirement pour atteindre la héro pour cela nous allons utiliser la librairie random qui génère de chiffre aléatoire ce qui nous permet d'associer cela à des déplacement.
Au début les déplacement était de la "téléportation" c'est à dire que la position du héro était traité avec des rafraichissement d'images c'est à dire, quand le joueur appuie sur la touche déplacer le héro change de position et donc une nouvelle image et créer sur cette position et l'autre est supprimer, mais cela pose un problème, quand la vitesse du héro est trop grande on voit les différente images des différentes position du héro, ....... suite .......
	
	\subsection{Tir}
Généralement dans un manic shooter le héro doit posséder un tir qui va détruire les ennemies 
	\subsection{collisions}
Les personnages sont confronté à des collisions qui doivent être gérer, c'est à dire quand le tir va aller

\section{Elements technique}

\subsection{Python}

pour ce projet nous utilisons python 3 avec 10 librairie externe à python :
OS,sys,pygame,tkinter,random,json,time,sympy

\subsection{Choix de l'interface graphique}

le choix des utilitaires utilisés peut différer le rendu du projet nous avons donc dû tester au fur et à mesure ce qui aller nous convenir le mieux

\subsection{Tkinter}
tkinter est une interface graphique qui est orienté logiciel elle permet la création de menu beaucoup plus simple dans notre cas cela nous sert pour le premier menu de notre jeu, du faite que les boutons sont directement
implémentés dans la librairie de tkinter.
 \begin{figure*}[ht!]
 \centering
 \includegraphics[width=0.7\linewidth]{tkinter.png}
 \caption{notre premier menu avec tkinter}
 \label{fig::example::one}
\end{figure*}

\href{https://www.tutorialspoint.com/python/python_gui_programming.htm}{source tkinter}

\subsection{Pygame}
Pygame est également une interface graphique qui est plus destiné au jeu,
dans notre cas nous l'utilisons pour l'intégralité de notre gameplay , les déplacement, les collisions , la carte et le choix de cette interface graphique nous est venu naturellement car elle offre une possibilité de déplacement intéressante.
 \begin{figure*}[ht!]
 \centering
 \includegraphics[width=0.5\linewidth]{diag.png}
 \caption{Gameplay avec pygame}
 \label{fig::example::one}
\end{figure*}

\href{http://www.pygame.org/docs/}{source pygame}

\section{Expérimentation et Usage}
capture d'écran rendu final / perfomance 

\subsection{Expérimentation}
Une phase d'expérimentation à du être nécessaire pour identifier ce qui peut poser problème.
Pour commencer nous avons penser initialiser une grille pour la carte et des personnage étant des numéro pouvait traverser la grille, nous nous sommes vite rendu compte que cela aller poser problème cela ne rendait pas le jeu dynamique.
Une fois que nous avons ouvert la fenêtre de pygame nous nous somme rendu compte que la grille aller être cette fenêtre et que les déplacements aller ce faire plus dynamiquement de cette manière nous avons donc oublier la grille.
Le choix des interface graphique c'est fait en essayant dans un premier temps tkinter à était solliciter et nous somme rendu compte que le rendu graphique n'était pas qualité, les menu ce faisait facilement donc le premier menu c'est fait avec celui ci mais pour la conception du jeu c'est tkinter qui à était retenu.

\section{Conclusion}
\end{document}
